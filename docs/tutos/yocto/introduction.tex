%===================================================%
%===================================================%
\chapter{Introduction}
%===================================================%
%===================================================%

The following work is being done on Ubuntu 14.04 LTS 64-bit.
The list of installed packages on the computer can be found at the following link. \framebox{\url{https://goo.gl/33UmDc}}.
\\
%%(\url{https://drive.google.com/open?id=0B1qotcVC6wOdOVVXcVRKcVBsd00})
%same link than the previous one but a bit long (in case if the shortened URL does not work)

To install the same packages on your VM or machine, you can use the following command :

\begin{lstlisting}[language=bash]
for pkg in $(cat install.list); do sudo apt-get -y install $pkg ; done
\end{lstlisting}

\begin{lstlisting}[language=bash]
sudo apt-get install aptitude && \
sudo aptitude install $(cat install.list | awk '{print $1}')
\end{lstlisting}

(This may take a while !)\vspace{1em}

\textbf{Interesting links  :}\vspace{1em}

\textbf{Yocto configuration}


\framebox{\url{http://picozed.org/content/building-zedboard-images}}.
\vspace{1em}

\textbf{meta-xilinx git}

\framebox{\url{https://github.com/Xilinx/meta-xilinx}}. \vspace{1em}

\textbf{Wiki Xilinx}

\framebox{\url{http://www.wiki.xilinx.com/Prepare+Boot+Image}}.\vspace{1em}

\textbf{Yocto tutorials}

\framebox{\url{http://wiki.elphel.com/index.php?title=Yocto_tests}}.
\vspace{1em}

\framebox{\url{http://git.yoctoproject.org/cgit/cgit.cgi/meta-xilinx/tree/README}}.

\vspace{1em}

\texttt{\framebox{yocto/meta-xilinx/docs/BOOT.sdcard}} (This directory can be found on meta-xilinx git)


Yocto provides tools and metadata for creating custom embedded systems with following main features :
\begin{itemize}
\item Images are tailored to specific hardware and use cases
\item But metadata is generally arch-independent
\item Unlike a distro, 'kitchen sink' is not included (we know what we need in advance)
\end{itemize}

Other keywords and their meaning is explained here.
\begin{itemize}
\item An image is a collection of 'baked' recipes (packages)
\item A 'recipe' is a set of instructions for building 'packages'
\begin{itemize}[label=$\star$]
\item Where to get the source and which patches to apply
\item Dependencies (on libraries or other recipes, for example)
\item Config/compile options, 'install' customization
\end{itemize}
\item A 'layer' is a logical collection of recipes representing the core, a board support package (BSP), or an application stack
\end{itemize}

%===================================================%
\section{Download sources from git repositories}
%===================================================%

Jethro is the last branch of git repositories.

Download Yocto meta layer:

\framebox{\texttt{git clone -b jethro git://git.yoctoproject.org/poky.git}}.

Download Xilinx meta layer:

\framebox{\texttt{git clone -b jethro git://github.com/Xilinx/meta-xilinx}}.

Download meta-openembedded layer:

\framebox{\texttt{git clone -b jethro git://github.com/openembedded/meta-openembedded}}.

Download openembedded-core :

\framebox{\texttt{git clone -b jethro git://github.com/openembedded/openembedded-core}}.



%===================================================%
\section{Build configuration}
%===================================================%

Once the packages are cloned, we can start modifying it.
Initialize a build using the 'oe-init-build-env' script in Poky.
\begin{lstlisting}[language=bash]
$ source oe-init-build-env
\end{lstlisting}

Once initialized configure bblayers.conf by adding the 'meta-xilinx' and meta-openembedded layers.
\texttt{
	BBLAYERS ?= " \
		<path to layer>/openembedded-core/meta \ \\
		<path to layer>/meta-xilinx \ \\
	    <path to layer>/meta-openembedded/meta-oe \ \\
		"
} 

To build a specific target BSP configure the associated machine in local.conf:
\texttt{
	MACHINE ?= "zedboard-zynq7"
}

Change 	\texttt{PACKAGE\_CLASSES '= "package\_rpm"} by

\texttt{PACKAGE\_CLASSES '= "package\_deb"}. Then, add the following lines :

\texttt{
IMAGE\_FEATURES += "package-management"
BB\_NUMBER\_THREADS = "nb" \\
PARALLEL\_MAKE = "-j nb"\\
DISTRO\_HOSTNAME = "zynq"\\
}
where nb = 2 x number of cores \\


Build the target file system image using bitbake:
\begin{lstlisting}[language=bash]
	$ bitbake core-image-minimal
\end{lstlisting}
Once complete the images for the target machine will be available in the output directory \framebox{\texttt{poky/build/tmp/deploy/images/}}.


\section{Xilinx SDK installation}

\begin{lstlisting}[language=bash]
$ sudo apt-get install ia32-libs
$ export CROSS\_COMPILE=arm-xilinx-linux-gnueabi-
$ source <Xilinx Tools installation directory>/ISE\_DS/settings64.sh
\end{lstlisting}

